\documentclass[]{tufte-handout}

% ams
\usepackage{amssymb,amsmath}

\usepackage{ifxetex,ifluatex}
\usepackage{fixltx2e} % provides \textsubscript
\ifnum 0\ifxetex 1\fi\ifluatex 1\fi=0 % if pdftex
  \usepackage[T1]{fontenc}
  \usepackage[utf8]{inputenc}
\else % if luatex or xelatex
  \makeatletter
  \@ifpackageloaded{fontspec}{}{\usepackage{fontspec}}
  \makeatother
  \defaultfontfeatures{Ligatures=TeX,Scale=MatchLowercase}
  \makeatletter
  \@ifpackageloaded{soul}{
     \renewcommand\allcapsspacing[1]{{\addfontfeature{LetterSpace=15}#1}}
     \renewcommand\smallcapsspacing[1]{{\addfontfeature{LetterSpace=10}#1}}
   }{}
  \makeatother

\fi

% graphix
\usepackage{graphicx}
\setkeys{Gin}{width=\linewidth,totalheight=\textheight,keepaspectratio}

% booktabs
\usepackage{booktabs}

% url
\usepackage{url}

% hyperref
\usepackage{hyperref}

% units.
\usepackage{units}


\setcounter{secnumdepth}{-1}

% citations

% pandoc syntax highlighting
\usepackage{color}
\usepackage{fancyvrb}
\newcommand{\VerbBar}{|}
\newcommand{\VERB}{\Verb[commandchars=\\\{\}]}
\DefineVerbatimEnvironment{Highlighting}{Verbatim}{commandchars=\\\{\}}
% Add ',fontsize=\small' for more characters per line
\newenvironment{Shaded}{}{}
\newcommand{\KeywordTok}[1]{\textcolor[rgb]{0.00,0.44,0.13}{\textbf{#1}}}
\newcommand{\DataTypeTok}[1]{\textcolor[rgb]{0.56,0.13,0.00}{#1}}
\newcommand{\DecValTok}[1]{\textcolor[rgb]{0.25,0.63,0.44}{#1}}
\newcommand{\BaseNTok}[1]{\textcolor[rgb]{0.25,0.63,0.44}{#1}}
\newcommand{\FloatTok}[1]{\textcolor[rgb]{0.25,0.63,0.44}{#1}}
\newcommand{\ConstantTok}[1]{\textcolor[rgb]{0.53,0.00,0.00}{#1}}
\newcommand{\CharTok}[1]{\textcolor[rgb]{0.25,0.44,0.63}{#1}}
\newcommand{\SpecialCharTok}[1]{\textcolor[rgb]{0.25,0.44,0.63}{#1}}
\newcommand{\StringTok}[1]{\textcolor[rgb]{0.25,0.44,0.63}{#1}}
\newcommand{\VerbatimStringTok}[1]{\textcolor[rgb]{0.25,0.44,0.63}{#1}}
\newcommand{\SpecialStringTok}[1]{\textcolor[rgb]{0.73,0.40,0.53}{#1}}
\newcommand{\ImportTok}[1]{#1}
\newcommand{\CommentTok}[1]{\textcolor[rgb]{0.38,0.63,0.69}{\textit{#1}}}
\newcommand{\DocumentationTok}[1]{\textcolor[rgb]{0.73,0.13,0.13}{\textit{#1}}}
\newcommand{\AnnotationTok}[1]{\textcolor[rgb]{0.38,0.63,0.69}{\textbf{\textit{#1}}}}
\newcommand{\CommentVarTok}[1]{\textcolor[rgb]{0.38,0.63,0.69}{\textbf{\textit{#1}}}}
\newcommand{\OtherTok}[1]{\textcolor[rgb]{0.00,0.44,0.13}{#1}}
\newcommand{\FunctionTok}[1]{\textcolor[rgb]{0.02,0.16,0.49}{#1}}
\newcommand{\VariableTok}[1]{\textcolor[rgb]{0.10,0.09,0.49}{#1}}
\newcommand{\ControlFlowTok}[1]{\textcolor[rgb]{0.00,0.44,0.13}{\textbf{#1}}}
\newcommand{\OperatorTok}[1]{\textcolor[rgb]{0.40,0.40,0.40}{#1}}
\newcommand{\BuiltInTok}[1]{#1}
\newcommand{\ExtensionTok}[1]{#1}
\newcommand{\PreprocessorTok}[1]{\textcolor[rgb]{0.74,0.48,0.00}{#1}}
\newcommand{\AttributeTok}[1]{\textcolor[rgb]{0.49,0.56,0.16}{#1}}
\newcommand{\RegionMarkerTok}[1]{#1}
\newcommand{\InformationTok}[1]{\textcolor[rgb]{0.38,0.63,0.69}{\textbf{\textit{#1}}}}
\newcommand{\WarningTok}[1]{\textcolor[rgb]{0.38,0.63,0.69}{\textbf{\textit{#1}}}}
\newcommand{\AlertTok}[1]{\textcolor[rgb]{1.00,0.00,0.00}{\textbf{#1}}}
\newcommand{\ErrorTok}[1]{\textcolor[rgb]{1.00,0.00,0.00}{\textbf{#1}}}
\newcommand{\NormalTok}[1]{#1}

% longtable

% multiplecol
\usepackage{multicol}

% strikeout
\usepackage[normalem]{ulem}

% morefloats
\usepackage{morefloats}


% tightlist macro required by pandoc >= 1.14
\providecommand{\tightlist}{%
  \setlength{\itemsep}{0pt}\setlength{\parskip}{0pt}}

% title / author / date
\title{Git e Github}
\author{Sérgio Carvalho}
\date{15 outubro, 2018}


\begin{document}

\maketitle




================================================================================

\begin{Shaded}
\begin{Highlighting}[]
\CommentTok{# "cerulean", "journal", "flatly", "readable", "spacelab", }
\CommentTok{# "united", "cosmo", "lumen", "paper", "sandstone", "simplex", "yeti"}

\CommentTok{# readable,}

    \CommentTok{# html_document:}
    \CommentTok{# theme: sandstone}
    \CommentTok{# highlight: zenburn }
    \CommentTok{# code_folding: hide}
    \CommentTok{# after_body: doc_suffix.html}
    \CommentTok{# before_body: doc_prefix.html}
    \CommentTok{# fig_caption: yes}
    \CommentTok{# fig_height: 4}
    \CommentTok{# fig_width: 10}
    \CommentTok{# df_print: paged}
    \CommentTok{# number_sections: yes}
    \CommentTok{# toc: yes}
    \CommentTok{# toc_float:}
    \CommentTok{#   collapsed: yes}
    \CommentTok{#   smooth_scroll: yes}
\end{Highlighting}
\end{Shaded}

\section{O que é o Git e o Github ?}\label{o-que-e-o-git-e-o-github}

\subsection{O Git}\label{o-git}

\begin{itemize}
\item
  É um sistema de controle de versão distribuído gratuito e de código
  aberto projetado para lidar com tudo, de projetos pequenos a muito
  grandes, com velocidade e eficiência, além de ser fácil de aprender
  Ele usa recursos como ramificação local barata, áreas de preparação
  convenientes e vários fluxos de trabalho.
\item
  Interessante: A microsoft migrou o seu sistema de controle de versão
  interna para o Git com o Github, pois pra o windows há cerca de 4 mil
  desenvolvedores em todo mundo e o controle de versão da microsoft não
  estava dando conta.
\end{itemize}

\subsection{O Github}\label{o-github}

\begin{itemize}
\tightlist
\item
  É um conjunto de repositórios publicados no Git e na internet, ou
  seja, além de o usuário poder fazer o controle de versão na máquina
  com o GitHub ele também pode fazer esse controle de versão online em
  um único repositório protegido por usuário e senha, isso é o Github,
  ou seja, é um local na internet onde estão os diversos repositórios do
  Git em todo mundo.
\end{itemize}

\section{Aprenda git em:}\label{aprenda-git-em}

\begin{itemize}
\tightlist
\item
  (GitHub, n.d.)
\item
  (Udemy, n.d.)
\item
  (GitBranching, n.d.)
\item
  (GitLab, n.d.)
\item
  (Git-try-io, n.d.)
\end{itemize}

\section{Configure a ferramenta}\label{configure-a-ferramenta}

Configure informações de usuário para todos os repositórios locais, como
o nome e o email que você quer ligado às suas transações de commit.

\begin{itemize}
\item
  \$ git config --global user.name ``{[}nome{]}''
\item
  \$ git config --global user.email ``{[}endereco-de-email{]}''
\end{itemize}

\section{Crie um repositório ou obtenha de uma URL
existente}\label{crie-um-repositorio-ou-obtenha-de-uma-url-existente}

\begin{itemize}
\item
  \$ git init {[}nome-do-projeto{]}
\item
  \$ git clone {[}url{]}
\end{itemize}

\section{Mude e remova os arquivos
versionados}\label{mude-e-remova-os-arquivos-versionados}

Remove o arquivo do diretório de trabalho e o prepara a remoção

\begin{itemize}
\tightlist
\item
  \$ git rm {[}arquivo{]}
\end{itemize}

Remove o arquivo do controle de versão mas preserva o arquivo localmente

\begin{itemize}
\tightlist
\item
  \$ git rm --cached {[}arquivo{]}
\end{itemize}

Muda o nome do arquivo e o prepara para o commit

\begin{itemize}
\tightlist
\item
  \$ git mv {[}arquivo-original{]} {[}arquivo-renomeado{]}
\end{itemize}

\section{Suprima o monitoramento}\label{suprima-o-monitoramento}

Ignore arquivos e diretórios temporários

\begin{itemize}
\tightlist
\item
  *.log
\item
  build/
\item
  temp-*
\end{itemize}

Um arquivo de texto chamado .gitignore suprime o versionamento acidental
de arquivos e diretórios correspondentes aos padrões especificados

\begin{itemize}
\tightlist
\item
  \$ git ls-files --others --ignored --exclude-standard
\end{itemize}

Lista todos os arquivos ignorados neste projeto

\section{Salve fragmentos}\label{salve-fragmentos}

Arquive e restaure mudanças incompletas.

Armazena temporariamente todos os arquivos monitorados modificados

\begin{itemize}
\tightlist
\item
  \$ git stash
\end{itemize}

Restaura os arquivos recentes em stash

\begin{itemize}
\tightlist
\item
  \$ git stash pop
\end{itemize}

Lista todos os conjuntos de alterações em stash

\begin{itemize}
\tightlist
\item
  \$ git stash list
\end{itemize}

Descarta os conjuntos de alterações mais recentes em stash

\begin{itemize}
\tightlist
\item
  \$ git stash drop
\end{itemize}

\section*{Referências}\label{referencias}
\addcontentsline{toc}{section}{Referências}

\hypertarget{refs}{}
\hypertarget{ref-Gittryio}{}
Git-try-io, Team. n.d. ``Git-Try-Io:'' \url{http://try.github.io/}.

\hypertarget{ref-GitBranching}{}
GitBranching, Team. n.d. ``GitBranching:''
\url{https://learngitbranching.js.org/}.

\hypertarget{ref-Gitbook}{}
GitHub, Team. n.d. ``Gitbook:'' \url{https://git-scm.com/book/en/v2}.

\hypertarget{ref-GitLab}{}
GitLab, Team. n.d. ``GitLab:'' \url{https://lab.github.com/courses}.

\hypertarget{ref-UdemyGit}{}
Udemy, Team. n.d. ``UdemyGit:''
\url{https://www.udemy.com/git-e-github-para-iniciantes/}.



\end{document}
